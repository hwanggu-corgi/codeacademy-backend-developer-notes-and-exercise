\documentclass[12pt]{article}
\usepackage[margin=2.5cm]{geometry}
\usepackage{enumerate}
\usepackage{amsfonts}
\usepackage{amsmath}
\usepackage{fancyhdr}
\usepackage{amsmath}
\usepackage{amssymb}
\usepackage{amsthm}
\usepackage{mdframed}
\usepackage{graphicx}
\usepackage{subcaption}
\usepackage{adjustbox}
\usepackage{listings}
\usepackage{xcolor}
\usepackage{courier}
\usepackage[utf]{kotex}
\usepackage{hyperref}
\usepackage{soul}
\usepackage{cancel}


\definecolor{codegreen}{rgb}{0,0.6,0}
\definecolor{codegray}{rgb}{0.5,0.5,0.5}
\definecolor{codepurple}{rgb}{0.58,0,0.82}
\definecolor{backcolour}{rgb}{0.95,0.95,0.92}

\lstdefinestyle{mystyle}{
    backgroundcolor=\color{backcolour},
    commentstyle=\color{codegreen},
    keywordstyle=\color{magenta},
    numberstyle=\tiny\color{codegray},
    stringstyle=\color{codepurple},
    basicstyle=\ttfamily\footnotesize,
    breakatwhitespace=false,
    breaklines=true,
    captionpos=b,
    keepspaces=true,
    numbers=left,
    numbersep=5pt,
    showspaces=false,
    showstringspaces=false,
    showtabs=false,
    tabsize=1
}

\lstset{style=mystyle}

\pagestyle{fancy}
\renewcommand{\headrulewidth}{0.4pt}
\lhead{Codeacademy}
\rhead{Notes}

\begin{document}
\title{Codeacademy Notes}

\section{Build a Back-End with Node/Express.js}
\subsection{Introduction}
\subsection{Node REPL}
\begin{itemize}
    \item Is an abbrebivation for Read-eval-print loop
    \item Node comes with built-in javascript REPL
    \item \texttt{.editor} goes into editor mode
    \begin{itemize}
        \item Use \texttt{CTRL + D} when ready to evaluate the input
    \end{itemize}
    \item A REPL can be extremely useful for performing calculations
    \item The Node environment contains a number of Node-specific \texttt{global}
    elements in addition to those built into the JavaScript language
    \begin{itemize}
        \item can be examined using command \texttt{console.log(global)}
    \end{itemize}
\end{itemize}

\subsection{Running a Program with Node}
\begin{itemize}
    \item Done using command \texttt{node myProgram.js}
    \item Javascript code is written to file \texttt{.js} extension
\end{itemize}

\subsection{Accessing the Process Object}
\begin{itemize}
    \item Node has a global \texttt{process} object with useful methods and information about the current process.
    \begin{itemize}
        \item \texttt{process.env} property is an object which stores and controls information about the environment in which the process is currently running
        \begin{itemize}
            \item \texttt{PWD} - holds a string with the directory where the current process is located
            \item \texttt{NODE\_ENV} - holds a value of either production or development

            \bigskip

            \underline{\textbf{Example}}

    \begin{lstlisting}
    if (process.env.NODE_ENV === 'development'){
        console.log('Testing! Testing! Does everything work?');
    }
    \end{lstlisting}

            \item \texttt{process.memoryUsage()} returns information on the CPU demands of the current process.
        \end{itemize}
    \end{itemize}
\end{itemize}

\end{document}
