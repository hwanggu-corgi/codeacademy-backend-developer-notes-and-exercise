\documentclass[12pt]{article}
\usepackage[margin=2.5cm]{geometry}
\usepackage{enumerate}
\usepackage{amsfonts}
\usepackage{amsmath}
\usepackage{fancyhdr}
\usepackage{amsmath}
\usepackage{amssymb}
\usepackage{amsthm}
\usepackage{mdframed}
\usepackage{graphicx}
\usepackage{subcaption}
\usepackage{adjustbox}
\usepackage{listings}
\usepackage{xcolor}
\usepackage{courier}
\usepackage[utf]{kotex}
\usepackage{hyperref}
\usepackage{soul}
\usepackage{cancel}


\definecolor{codegreen}{rgb}{0,0.6,0}
\definecolor{codegray}{rgb}{0.5,0.5,0.5}
\definecolor{codepurple}{rgb}{0.58,0,0.82}
\definecolor{backcolour}{rgb}{0.95,0.95,0.92}

\lstdefinestyle{mystyle}{
    backgroundcolor=\color{backcolour},
    commentstyle=\color{codegreen},
    keywordstyle=\color{magenta},
    numberstyle=\tiny\color{codegray},
    stringstyle=\color{codepurple},
    basicstyle=\ttfamily\footnotesize,
    breakatwhitespace=false,
    breaklines=true,
    captionpos=b,
    keepspaces=true,
    numbers=left,
    numbersep=5pt,
    showspaces=false,
    showstringspaces=false,
    showtabs=false,
    tabsize=1
}

\lstset{style=mystyle}

\pagestyle{fancy}
\renewcommand{\headrulewidth}{0.4pt}
\lhead{Codeacademy}
\rhead{Notes}

\begin{document}
\title{Codeacademy Notes}

\section{Navigation}
\subsection{File System}

\subsection{ls}
\begin{itemize}
    \item looks at the directory you are in
    \item lists all the files and directories inside of it
\end{itemize}

\subsection{pwd}
\begin{itemize}
    \item stands for “print working directory.”
    \item prints the path of directory you are currently in
    \item Together \texttt{ls}, the \texttt{pwd} are useful to show where you are in the filesystem.
\end{itemize}

\subsection{cd}
\begin{itemize}
    \item stands for “change directory.”
    \item switches you into the directory you specify.

    \bigskip

    \underline{\textbf{Example}}

    \texttt{cd 2015}

    \begin{itemize}
        \item 2015 is called argument
    \end{itemize}
    \item \texttt{..} is used to move up directory
    \bigskip

    \underline{\textbf{Example}}

    \texttt{cd ..}
\end{itemize}

\subsection{mkdir}
\begin{itemize}
    \item stands for “make directory”
    \item takes in a directory name as an argument and then creates a new directory in the current working directory.
\end{itemize}

\subsection{touch}
\begin{itemize}
    \item creates a new file inside the working directory.
    \item takes in a filename as an argument and then creates an empty file with that name in the current working directory.
\end{itemize}

\subsection{Helper Commands}
\begin{itemize}
    \item \texttt{clear} - clears terminal
    \item \texttt{tab} - autocomplete your command
    \item up down arrows - cycle through previous commands
    \begin{itemize}
        \item down arrow - most recent command
    \end{itemize}
\end{itemize}

\subsection{Review}
\begin{itemize}
    \item the \texttt{command line} is the interface for computer's operating system
    \item A \texttt{filesystem} organizes a computer’s files and directories into a tree structure
    \item Naviating through directory
    \begin{itemize}
        \item \texttt{pwd} outputs the name of the current working directory.
        \item \texttt{ls} lists all files and directories in the working directory.
        \item \texttt{cd} switches you into the directory you specify.
        \item \texttt{mkdir} creates a new directory in the working directory.
        \item \texttt{touch} creates a new file inside the working directory.
    \end{itemize}
\end{itemize}

\subsection{Quiz}
\begin{itemize}
    \item What does the following command do?
    \begin{enumerate}[a)]
        \item It creates a file named popular.txt in the media directory.
        \item It creates a file named popular.txt in your working directory.
        \item It changes the working directory to the media directory.
        \item This command is not formed correctly.
    \end{enumerate}

    \bigskip

    \textbf{Answer:} A

    \item If the current working directory is home/, which of the following commands
    will navigate to the movies/ directory in the tree below?

    \begin{enumerate}[a)]
        \item cd movies
        \item mkdir media/movies
        \item ls media/movies
        \item cd media/movies
    \end{enumerate}

    \bigskip

    \textbf{Answer:} D

    \item How do you print the current working directory?

    \begin{enumerate}[a)]
        \item mkdir
        \item pwd
        \item cd
        \item ls
    \end{enumerate}

    \bigskip

    \textbf{Answer:} B

    \item How would you change to one directory above the current working directory?

    \begin{enumerate}[a)]
        \item cd ..
        \item cd ../..
        \item ls ..
        \item mkdir ..
    \end{enumerate}

    \bigskip

    \textbf{Answer:} A

    \item What is a filesystem?
    \begin{enumerate}[a)]
        \item It organizes a computer’s files and directories into a tree structure.
        \item It’s a type of directory.
        \item It’s a text interface with a computer.
        \item It is a directive to the computer to perform a specific task.
    \end{enumerate}

    \bigskip

    \textbf{Answer:} A

    \item What is a directory?
    \begin{enumerate}[a)]
        \item A file
        \item A command to a computer
        \item A tree structure
        \item A folder used to store files
    \end{enumerate}

    \bigskip

    \textbf{Answer:} D

    \item How would you create a file named text.txt in your current directory?
    \begin{enumerate}[a)]
        \item ls text.txt
        \item mkdir text.txt
        \item touch text.txt
        \item touch home/text.txt
    \end{enumerate}

    \bigskip

    \textbf{Answer:} C
\end{itemize}

\section{Project: Bicycle world}
\subsection{Task Solution}
\begin{enumerate}[1.]
    \item \texttt{pwd}
    \item \texttt{ls}
    \item \texttt{cd freight/}
    \item \texttt{ls}
    \item \texttt{cd porteur/}
    \item \texttt{pwd ../..} \texttt{ls}
    \item \texttt{cd mountain/downhill}
    \item \texttt{touch dirt.txt}
    \item \texttt{touch mud.txt}
    \item \texttt{ls}
    \item \texttt{mkdir safety}
    \item \texttt{cd ../../}
    \item \texttt{ls}
    \item \texttt{mkdir bmx}
    \item \texttt{touch bmx/tricks.txt}
    \item \texttt{ls}
\end{enumerate}

\end{document}
