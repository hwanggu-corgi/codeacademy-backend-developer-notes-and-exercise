\documentclass[12pt]{article}
\usepackage[margin=2.5cm]{geometry}
\usepackage{enumerate}
\usepackage{amsfonts}
\usepackage{amsmath}
\usepackage{fancyhdr}
\usepackage{amsmath}
\usepackage{amssymb}
\usepackage{amsthm}
\usepackage{mdframed}
\usepackage{graphicx}
\usepackage{subcaption}
\usepackage{adjustbox}
\usepackage{listings}
\usepackage{xcolor}
\usepackage{courier}
\usepackage[utf]{kotex}
\usepackage{hyperref}
\usepackage{soul}
\usepackage{cancel}


\definecolor{codegreen}{rgb}{0,0.6,0}
\definecolor{codegray}{rgb}{0.5,0.5,0.5}
\definecolor{codepurple}{rgb}{0.58,0,0.82}
\definecolor{backcolour}{rgb}{0.95,0.95,0.92}

\lstdefinestyle{mystyle}{
    backgroundcolor=\color{backcolour},
    commentstyle=\color{codegreen},
    keywordstyle=\color{magenta},
    numberstyle=\tiny\color{codegray},
    stringstyle=\color{codepurple},
    basicstyle=\ttfamily\footnotesize,
    breakatwhitespace=false,
    breaklines=true,
    captionpos=b,
    keepspaces=true,
    numbers=left,
    numbersep=5pt,
    showspaces=false,
    showstringspaces=false,
    showtabs=false,
    tabsize=1
}

\lstset{style=mystyle}

\pagestyle{fancy}
\renewcommand{\headrulewidth}{0.4pt}
\lhead{Codeacademy}
\rhead{Notes}

\begin{document}
\title{Codeacademy Notes}

\section{Navigation}
\subsection{File System}

\subsection{ls}
\begin{itemize}
    \item looks at the directory you are in
    \item lists all the files and directories inside of it
\end{itemize}

\subsection{pwd}
\begin{itemize}
    \item stands for “print working directory.”
    \item prints the path of directory you are currently in
    \item Together \texttt{ls}, the \texttt{pwd} are useful to show where you are in the filesystem.
\end{itemize}

\subsection{cd}
\begin{itemize}
    \item stands for “change directory.”
    \item switches you into the directory you specify.

    \bigskip

    \underline{\textbf{Example}}

    \texttt{cd 2015}

    \begin{itemize}
        \item 2015 is called argument
    \end{itemize}
    \item \texttt{..} is used to move up directory
    \bigskip

    \underline{\textbf{Example}}

    \texttt{cd ..}
\end{itemize}

\subsection{mkdir}
\begin{itemize}
    \item stands for “make directory”
    \item takes in a directory name as an argument and then creates a new directory in the current working directory.
\end{itemize}

\subsection{touch}
\begin{itemize}
    \item creates a new file inside the working directory.
    \item takes in a filename as an argument and then creates an empty file with that name in the current working directory.
\end{itemize}

\end{document}
